\makeatletter\def\input@path{{images/}{styles/}{corps/}{bib/}}\makeatother

\documentclass{transp_java}
\usepackage{etex}
\title[UE 1.1 : Introduction \`{a} la programmation - Python 3]{CM. R\'ecursivit\'e}
\author{A. Malek TOUMI\\ toumiab@ensta-bretagne.fr}
%\date[05/04/2015]{Toumi A. Malek.  Python 3.4}
\date{2018/2019}
\institute{ENSTA Bretagne}


%\usetheme{Transp}
\usetheme{progressbar}
%\usetheme{Darmstadt}
\progressbaroptions{imagename= python-HE.png}



\mode<handout>{
  \usecolortheme{dove}
  \usepackage{pgfpages}
  \pgfpagesuselayout{4 on 1}[a4paper, border shrink=5mm, landscape]
}



\begin{document}

\frame{\titlepage}

\AtBeginSection[]
{
  \begin{frame}<beamer>
    \frametitle{Sommaire}
    \tableofcontents[current,currentsection]
  \end{frame}
}

\AtBeginSubsection[]
{
  \begin{frame}<beamer>
    \frametitle{Sommaire}
    \tableofcontents[current,currentsubsection]
  \end{frame}
}

\newlength\dunder
\settowidth{\dunder}{\_}
\newcommand{\twound}{\rule{2\dunder}{0.4pt}}


%%%

\section{Rappel}
\subsection{Param�tres formels/param�tres effectifs}

\begin{frame}
  \frametitle{Fonction fact}%\chrono}

  \begin{itemize}[<+-|alert@+>]
  \item Param�tres formels
  \item Param�tres effectifs
  \end{itemize}

  \begin{semiverbatim}
    def \textcolor{blue}{fact}({\color<1>{red} nb}):

    \hspace{1cm}if(nb == 0):

    \hspace{1cm}...


    def cnp(n, p):

    \hspace{1cm}num = \textcolor{blue}{fact}({\color<2>{red}n})

    \hspace{1cm}...

  
  \end{semiverbatim}
\end{frame}
\subsection{Port�e de variable}
\begin{frame}[fragile]
  \frametitle{Variables globales/locales}%}
  
	\begin{remarques}
	\begin{itemize}[<+->]
		\item Les variables globales sont visibles dans les fonctions appel�es mais on ne peut les r�affecter
		\item Le contenu d'une variable globale est modifiable si elle est mutable (modifiable).
		\item Les variables locales d'une fonction ne sont pas visibles dans les niveaux sup�rieurs
		\item Les fonctions peuvent modifier des variables globales (les types non modifiables)  avec l'instruction \emph{global} 
		\item La fonction \emph{globals()} retourne le dictionnaire des objets (variables) globaux. \\
		\item La fonction \emph{locals()} retourne la liste des objets de l'espace local en cours
   \end{itemize}
  \end{remarques}
	
\end{frame}
\begin{frame}[fragile]
  \frametitle{Variables globales/locales}%}
\small{
 \begin{exemple}[Modification d'une variable globale en local]
    \begin{semiverbatim}
		\textcolor[rgb]{0,0,1}{def} incremCompt():  \textcolor[rgb]{0,0.58,0}{ # fonction sans param. d'entr�e}
    \only<1>{\textcolor[rgb]{0,0,1}{global} compteur  \textcolor[rgb]{0,0.58,0}{# d�finition var. globale}}
    compteur += 1
    print('Appel�', compteur, 'fois')

compteur = 0 \textcolor[rgb]{0,0.58,0}{# initialisation du compteur}
\only<1>{incremCompt() \textcolor[rgb]{0,0.58,0}{# => affiche: Appel� 1 fois} 
incremCompt() \textcolor[rgb]{0,0.58,0}{# => affiche: Appel� 2 fois}
incremCompt() \textcolor[rgb]{0,0.58,0}{# => affiche: Appel� 3 fois}}
\only<2-3>{incremCompt() \textcolor[rgb]{0,0.58,0}{# => affiche: Appel� 1 fois ? V/F?}  
incremCompt() \textcolor[rgb]{0,0.58,0}{# => affiche: Appel� 1 fois V/F?}
}
\only<3>{\color{red}{UnboundLocalError: local variable 'compteur' 
referenced before assignment}}
				
 \end{semiverbatim}




  \end{exemple}
	}
%}
\end{frame}


\begin{frame}[fragile]
  \frametitle{Variables globales/locales}%}
\small{
 \begin{exemple}[Modification d'une variable globale en local]
    \begin{semiverbatim}
		\textcolor[rgb]{0,0,1}{def} maville(lm):
    ville = "Brest" \# variable locale
    print (ville, cp,lm)
    \onslide<2->{test()   
\textcolor[rgb]{0,0,1}{def} test():
    print('test',ville)
}    

\textcolor[rgb]{0,0.58,0}{\# cp=29200 variable globale vue par la fct}
cp = 29200 \textcolor[rgb]{0,0.58,0}{\# variable globale}
maville('France') \textcolor[rgb]{0,0.58,0}{\# => affiche 'Brest 29200 France'}
\onslide<3>{\# => affiche 'test, Brest'	V/F? }
			
 \end{semiverbatim}
  \end{exemple}
	}
%}
\end{frame}

\section{R�cursivit�}


\subsection{D�finition}


\begin{frame}
  \frametitle{D�finition}%}
\begin{defi}[Fonction r�cursive]
  Une fonction ou une m�thode est dite r�cursive si elle se d�finit � partir
  d'elle m�me, c'est-�-dire si elle comporte \textbf{au moins} un appel � elle m�me dans son
  corps.
\end{defi}

\pause
\begin{alertblock}{Attention}
  �viter les boucles infinies (appel syst�matique � la fonction).
\end{alertblock}
\pause
\begin{exemple}[Fonction infinie]
  \begin{semiverbatim}
    def sansFin(n):

    \hspace{1cm}return n+sansFin(n-1)

   
  \end{semiverbatim}
\end{exemple}
\end{frame}


\begin{frame}
  \frametitle{Exemple}%}

  Traduction imm�diate des fonctions d�finies par r�currence.

  \begin{exemple}[Factorielle]
    $\left\{
      \begin{array}{l}
        \color<3>{red}0! = 1\\
        \color<4>{red}n! = n(n-1)!
      \end{array}
    \right.
    $\\
    
    \pause

    \begin{semiverbatim}
      def \textcolor{blue}{fact}(n):\pause

      \hspace{1cm}if n==0 : \only<5>{\textcolor[rgb]{0,0.58,0}{\# Condition d'arr�t}}

      \hspace{2cm}return 1\pause

      \hspace{1cm}else:

      \hspace{2cm}return n * \textcolor{blue}{fact}(n-1)

    \end{semiverbatim}
  \end{exemple}
\end{frame}

\begin{frame}
  \frametitle{Condition d'arr�t}%}

  \begin{defi}[Condition d'arr�t]
    Condition pour laquelle il n'y a pas d'appel r�cursif.
  \end{defi}
  \pause
  \begin{alertblock}{Important}
    Toute fonction r�cursive doit comporter au moins une condition
    d'arr�t. Sinon : boucle infinie.
  \end{alertblock}
  \pause
  \begin{remarque}
    Une fonction peut comporter plusieurs conditions d'arr�t.
  \end{remarque}
\end{frame}


\begin{frame}
  \frametitle{Condition d'arr�t}%}
  \begin{exemple}[Plusieurs conditions d'arr�t]
    \begin{semiverbatim}
      def fact(n):

      \hspace{1cm}{\color<2>{red}if (n==0):}
      
      \hspace{2cm}return 1

      \hspace{1cm}{\color<2>{red}elif (n==1):}

      \hspace{2cm}return 1

      \hspace{1cm}else :

      \hspace{2cm}return n * fact(n-1)

    \end{semiverbatim}
  \end{exemple}
\end{frame}


\subsection{Pile d'appel}

%\begin{frame}
  %\frametitle{Pile d'appel}
  %\begin{itemize}[<+->]
  %\item Zone de m�moire
  %\item Contient :
    %\begin{itemize}
    %\item<3-|alert@8-9> param�tres
    %\item<4-|alert@10> adresse de retour
    %\item<5-|alert@11> variables locales
		%\item<6-|alert@12> variables globales
    %\end{itemize}
  %\item<7-> Fonctionnement lors de l'appel d'une fonction
  %\end{itemize}
%
  %\uncover<8->{
  %\begin{tabular}{ll}
    %\begin{minipage}{0.8\linewidth}
    %\begin{semiverbatim}
      %def methode({\color<8>{red} a}, {\color<9>{red} b}):
%
      %\hspace{1cm}{\color<11>{red} i = 0}
			%\hspace{1cm} ...
			%
      %methode({\color<8>{red}2}, {\color<9>{red}1.5})
    %\end{semiverbatim}
  %\end{minipage}&
  %\begin{minipage}{0.15\linewidth}
  %\begin{tabular}{|r|}
    %\hline
    %\uncover<11>{i}\\
    %\hline
    %\uncover<10->{0x1E3C85}\\
    %\hline
    %\uncover<9->{1.5}\\
    %\hline
    %\uncover<8->{2}\\
    %\hline
  %\end{tabular}
  %\end{minipage}
%\end{tabular}
%}
%\end{frame}
%

\begin{frame}
  \frametitle{Pile et r�cursivit�}%}

  \begin{itemize}
  \item Pile d'appel : notion fondamentale pour la r�cursivit�.
  \item Principe : tous les appels sont empil�s, trait�s, puis d�pil�s.
  \end{itemize}	
		
		~\\

  \begin{tabular}{ll}
    \begin{minipage}{0.6\linewidth}
    \begin{semiverbatim}
      def fact(n) :

      \hspace{1cm}if n==0:

      \hspace{2cm}return 1

      \hspace{1cm}else:

      \hspace{2cm}return n*fact(n-1)


    \textcolor[rgb]{1,0,0}{>>>res = fact(2)}
    \end{semiverbatim}
  \end{minipage}&
  \begin{minipage}{0.35\linewidth}
  \begin{tabular}{|ll|}
    \hline
    \uncover<5-6>{fact(0)}&\uncover<6>{=1}\\
    \hline
    \uncover<3-7>{fact(1)}&\uncover<4-7>{=\only<-6>{1*fact(0)}\only<7->{1}}\\
    \hline
    fact(2)&\uncover<2->{=\only<-7>{2*fact(1)}\only<8->{2}}\\
    \hline
  \end{tabular}
  \end{minipage}
\end{tabular}

\end{frame}

\subsection{M�canisme des traitements}



\begin{frame}[fragile]
  \frametitle{Post et pr�-traitement}%}
 
 \begin{exemple}[Puissance d'entier : x**n]

  \begin{semiverbatim}\scriptsize{
	\vspace{-0.8cm}
	def puissance(x,n):
	    if n == 0:
	        \only<2->{print("cas de base n :", n)}
	        return 1 \uncover<2->{\textcolor[rgb]{0,0.58,0}{\# cas de base}}
	    else:
	        \only<2,3,5>{print("pr�traitement pour n :", n)} 
	        \only<1-3>{return}\only<4->{y =} x* puissance (x, n-1)
	        \only<4->{print("post-traitement n :", n)} 
	        \only<4->{return y}
	puissance(2,5)}
\textcolor[rgb]{0,0.58,0}{\# sortie}
\only<3,5>{\tiny{pretraitement de n :  5
pretraitement de n :  4
pretraitement de n :  3
pretraitement de n :  2
pretraitement de n :  1}}
\only<3-5>{\tiny{cas de base n : 0}}
\only<4-5>{\tiny{post-traitement de n: 1
post-traitement de n : 2
post-traitement de n : 3
post-traitement de n : 4
post-traitement de n : 5}}
  \end{semiverbatim}
	
\end{exemple}


\end{frame}
\subsection{Explosion combinatoire}

\begin{frame}
  \frametitle{Suite de Fibonacci}%}
  $\left\{
    \begin{array}{l}
      \color<2>{red}F_0 = 0\\
      \color<2>{red}F_1 = 1\\
      \color<3>{red}\forall n \geqslant 2, F_n = F_{n-1} + F_{n-2}
    \end{array}
  \right.
  $\\

  \only<2-3>{
    \begin{itemize}
    \item<2-> Deux conditions d'arr�t
    \item<3-> Une condition de r�currence
    \end{itemize}}
  
  \only<4->{
    \begin{semiverbatim}
      def \textcolor{blue}{fibo}(n):
      
      \hspace{1cm}if n==0:
      
      \hspace{2cm}return 0

      \hspace{1cm}elif n==1:

      \hspace{2cm}return 1

      \hspace{1cm}else

      \hspace{2cm}return \textcolor{blue}{fibo}(n-1) + \textcolor{blue}{fibo}(n-2)
    
    \end{semiverbatim}
  }
\end{frame}

\begin{frame}
  \frametitle{Explosion de la pile d'appels}%}
  �volution de la pile lors du calcul de F(3) :\\
  \begin{tabular}{|ll|}
    \hline
    \uncover<7-8>{\color<7-8>{purple}f(0)} & \uncover<8>{= 0}\\
    \hline
    \uncover<5-8,10-11>{\color<4-6>{purple}\color<10-11>{blue}f(1)} & \uncover<6-8,11>{= 1}\\
    \hline
    \uncover<3-11>{\color<3-9>{blue}f(2)} & \uncover<4-11>{= \only<-8>{{\color<4-6>{purple}f(1)}+{\color<7-9>{purple}f(0)}}\only<9->{1}}\\
    \hline
    f(3) & \uncover<2->{= \only<2-11>{{\color<2-9>{blue}f(2)}+{\color<10-11>{blue}f(1)}} \only<12->{2}}\\
    \hline
  \end{tabular}\\
	\footnotesize{
  \begin{itemize}
  \item<13-> Traduction imm�diate de la formule de r�currence
  \item[$\Rightarrow$]<14-> peu efficace dans ce cas
  \item<15-> Nombre d'appels � F en $\Theta(2^n)$
  \end{itemize}
	}
	\tiny{
	\uncover<16->{
	\begin{alertblock}{Important}
	 \begin{itemize}
		 \item <16-> Le nombre d'appels est limit� par d�faut, � 1000 appels. \\ $\Rightarrow$ \small{\emph{RuntimeError: maximum recursion depth exceeded ...}} 
		 \item <17> Exemple : Augmenter la taille �  1500: \texttt{sys.setrecursionlimit(1500)} 
	 \end{itemize}
		\end{alertblock}	}
}
  
\end{frame}


\subsection{Conception d'algorithme r�cursif}

\begin{frame}
  \frametitle{Conception d'algorithme r�cursif}%}

  \begin{itemize}[<+->]
  \item D�couper l'algorithme en �tapes
  \item D�terminer les r�gles de passage d'une �tape � l'autre
  \item[$\Rightarrow$] l'�tape $n$ d�pend de l'�tape $n-1$ (�ventuellement $n-2$)
  \item D�terminer les conditions d'arr�t
  \end{itemize}

  \pause
  \begin{remarque}
    Tout algorithme r�cursif peut s'�crire de mani�re it�rative, et
    r�ciproquement. 
  \end{remarque}
\end{frame}


\subsection{Exemples}

\begin{frame}
  \frametitle{Recherche par dichotomie}%}

  \begin{itemize}[<+->]
  \item Recherche d'un �l�ment dans un tableau tri�
  \item Principe :
    \begin{itemize}
    \item comparer l'�l�ment recherch� avec le milieu du tableau
    \item si $>$ : recherche dans le tableau $\Leftrightarrow$ recherche dans
      la partie droite
    \item si $<$ : recherche dans le tableau $\Leftrightarrow$ recherche dans
      la partie gauche
    \item si $=$ : �l�ment trouv� (condition d'arr�t)
    \item autre condition d'arr�t : taille du tableau  $\leqslant 1$
    \end{itemize}
  \end{itemize}
\end{frame}

\begin{frame}
  \frametitle{Recherche par dichotomie}%}
  \begin{semiverbatim}
    def rech(tab, val, deb, fin):

  \hspace{1cm}n = (deb + fin)//2 \pause

  \hspace{1cm}if tab[n] == val:

    \hspace{2cm}return True \pause

  \hspace{1cm}elif deb >= fin :

    \hspace{2cm}return False \pause

  \hspace{1cm}elif tab[n] > val:

    \hspace{2cm}return rech(tab, val, deb, n-1)

  \hspace{1cm}else:

    \hspace{2cm}return rech(tab, val, n+1, fin)

  \end{semiverbatim}
\end{frame}


\begin{frame}[fragile]
  \frametitle{Tours de Hano�}%}
  Principe :
  \begin{itemize}
  \item Disques empil�s : tour
  \item Ne jamais empiler un disque sur un disque plus petit
  \item D�placer un seul disque � la fois
  \end{itemize}
  \begin{center}
\tikzset{
    disc/.style={shade, shading=radial, rounded rectangle,minimum height=.3cm,
        inner color=#1!20, outer color=#1!60!gray},
    disc 1/.style={disc=yellow, minimum width=12mm},
    disc 2/.style={disc=orange, minimum width=16mm},
    disc 3/.style={disc=red, minimum width=20mm},
    disc 4/.style={disc=green, minimum width=24mm},
    disc 5/.style={disc=blue, minimum width=28mm},
    disc 6/.style={disc=purple, minimum width=32mm},
    disc 7/.style={disc=teal, minimum width=36mm},
}

% Define some colors, I don't like plain green and brown.
\definecolor{darkgreen}{rgb}{0.2,0.55,0}
\definecolor{darkbrown}{rgb}{0.375,0.25,0.125}

\begin{tikzpicture}
  \foreach \n/\x in {1/0cm,2/3.5cm,3/7cm} {
    \begin{scope}[xshift=\x]
      \fill[darkbrown] (-1.5cm, 0) rectangle (1.5cm,0.2cm)
      (-1mm,2mm) rectangle (1mm,3.2cm);
      \ifnum \n=1
      \foreach \y in {1,...,7} {
        \node[disc \y,yshift={32mm-\y*4mm}] {\scriptsize\y};
      }
      \fi
      % \expandafter\discs\csname pole \n\endcsname
    \end{scope}
  }
\end{tikzpicture}
   % \includegraphics[width=0.7\linewidth]{tours_hanoi}
  \end{center}
\end{frame}

\begin{frame}
  \frametitle{Tours de Hano� : r�solution}%}
  \begin{itemize}
  \item R�solution par une fonction r�cursive
  \item �tapes de l'algorithme : hauteur de la tour
  \item Hauteur 0 : �vident
  \item R�currence :
    \begin{itemize}
    \item on suppose savoir d�placer une tour de hauteur $h-1$
    \item on veut d�placer une tour de hauteur $h$ de A vers B
    \item d�placer les $h-1$ premiers �l�ments de A vers C
    \item d�placer l'�l�ment restant de A vers B
    \item d�placer les $h-1$ premiers �l�ments de C vers B
    \end{itemize}
  \end{itemize}
\end{frame}

\begin{frame}
  \frametitle{Tours de Hano� : algorithme}%}

%\begin{algorithme}{Algorithme des tours de Hano�}{alg:hanoi}
%  \proc \algverb{hanoi} (entier n, entier source, entier dest, entier tmp)\\
%  \vars \=\\
%  \> n : hauteur de la tour\\
%  \> source, dest, tmp\\
%  \debut \=\\
%  \> \si \= $n>0$ \alors\\
%  \> \> hanoi(n-1, source, tmp, dest)\\
%  \> \> d�placer(source, dest)\\
%  \> \> hanoi(n-1, tmp, dest, source)\\
%  \> \fin\\
%  \fin
%\end{algorithme}
  {\small 
\begin{procedure}[H]
  \caption{hanoi(entier n, entier source, entier dest, entier tmp)}
  \tcc{n~: hauteur de la tour\\ source, dest, tmp~: 
    position d'origine, finale et interm�diaire de la tour � d�placer}
  \If{$n>0$} { 
    \hanoi(n-1, source, tmp, dest) \;
    d�placer(source, dest) \;
    \hanoi(n-1, tmp, dest, source) \;
  }
\end{procedure}
% \begin{algorithmic}[1]
%     \Procedure{Hano�}{entier n, entier source, entier dest, entier tmp}
%     \Var
%     \State n : hauteur de la tour
%     \State source, dest, tmp
%     \EndVar
%     \If{$n>0$}
%     \State hanoi(n-1, source, tmp, dest)
%     \State d�placer(source, dest)
%     \State hanoi(n-1, tmp, dest, source)
%     \EndIf
%     \EndProcedure
%   \end{algorithmic}
}
\end{frame}

\begin{frame}
  \frametitle{�tude de l'algorithme tours de Hano�}%}
  $C(n)$ : nombre d'op�ration pour d�placer une tour de hauteur $n$\\
  \pause ~\\
    $\left\{
      \begin{array}{l}
        C(n+1) = 2C(n) + 1\\
        C(0) = 0
      \end{array}
    \right.
    $\\
    \pause ~\\
    R�solution : $C(n) = 2^n - 1$
    
    \vspace{2mm}\pause 
    L�gende des tours de Hano� : dans un temple Bouddhiste, des moines ont re�u
    pour mission de d�placer une tour de Hano� de 64 disques. Lorsqu'ils
    l'auront d�plac�e, le monde tombera en poussi�re.\\
    \pause
    $2^{64}-1 = 18\, 446\, 744\, 073\, 709\, 551\, 615$

    Un d�placement par seconde $\leadsto$ 580 milliards d'ann�es
\end{frame}


%\begin{frame}
  %\frametitle{Monnaie}
  %Probl�me : 
  %\begin{itemize}[<+->]
  %\item Combien de mani�res de rendre la monnaie sur une somme $s$
    %avec 1, 2 ou 5 \euro{} ?
  %\item R�solution : m�thodes r�cursives.
  %\end{itemize}
%
  %\pause Algorithme :
  %\begin{itemize}[<+->]
  %\item Rendre la monnaie sur $s$ avec des pi�ces de 1 \euro{} : 1 seule possibilit�
  %\item Rendre la monnaie sur $s$ avec 1 ou 2 \euro{} :
    %\begin{itemize}
    %\item Donner une pi�ce de 2\euro{} et rendre la monnaie sur $s-2$ avec 1 ou
      %2 \euro{}
    %\item Ou rendre la monnaie uniquement avec 1\euro{}
    %\end{itemize}
  %\item Rendre la monnaie sur $s$ avec 1, 2 ou 5 \euro{} : m�me principe
  %\end{itemize}
%\end{frame}
%
%\begin{frame}
  %\frametitle{Algorithme}
%%  \begin{algorithme}{Rendre la monnaie}{alg:monnaie}
%%    \proc \algverb{monnaie1} (somme s)\\
%%    \debut \=\\
%%    \> retourner 1\\
%%    \fin\\ \pause
%%    \proc \algverb{monnaie1-2} (somme s)\\
%%    \debut \=\\
%%    \> retourner monnaie1-2(s-2) + monnaie1(s)\\
%%    \fin\\ \pause
%%    \proc \algverb{monnaie1-2-5} (somme s)\\
%%    \debut \=\\
%%    \> retourner monnaie1-2-5(s-5) + monnaie1-2(s) + monnaie1(s)\\
%%    \fin
%%  \end{algorithme}
  %{\small 
%\begin{procedure}[H]
  %\caption{monnaie1(somme s)}
  %retourner 1\;
%\end{procedure}
%\begin{procedure}[H]
  %\caption{monnaie1-2(somme s)}
 %retourner \emph{monnaie1-2}(s-2) + \emph{monnaie1}(s)\;
%\end{procedure}
%\begin{procedure}[H]
  %\caption{monnaie1-2-5(somme s)}
 %retourner \emph{monnaie1-2-5}(s-5) + \emph{monnaie1-2}(s) + \emph{monnaie1}(s)\;
%\end{procedure}
%
%% \begin{algorithmic}[1]
%%     \Procedure{monnaie1}{somme s}
%%       \State retourner 1
%%     \EndProcedure
%%     \Procedure{monnaie1-2}{somme s}
%%       \State retourner monnaie1-2(s-2) + monnaie1(s)
%%     \EndProcedure
%%     \Procedure{monnaie1-2-5}{somme s}
%%       \State retourner monnaie1-2-5(s-5) + monnaie1-2(s) + monnaie1(s)
%%     \EndProcedure
%%   \end{algorithmic}
%}
%\end{frame}
%
%\begin{frame}
  %\frametitle{Algorithme : condition d'arr�t}
  %\begin{itemize}
  %\item Conditions d'arr�t de l'algorithme ?
  %\item \emph{monnaie1} non r�cursive : pas de condition d'arr�t
  %\item Pour \emph{monnaie1-2} : s'arr�ter si $s<2$
  %\item Pour \emph{monnaie1-2-5} : s'arr�ter si $s<5$
  %\end{itemize}
%\end{frame}
%
%\begin{frame}
  %\frametitle{Quicksort}%}
%\only<1-4>{
  %{\small 
%\begin{procedure}[H]
  %\caption{triSeg(tableau\_entier tab, entier debut, entier fin)}
  %\If{$debut<fin$} {\pause
      %choisir �l�ment pivot $p$ entre debut et fin \;\pause
      %placer pivot en $i$, $\leqslant$ pivot avant, $\geqslant$ pivot apr�s \;\pause
      %\triSeg(tab, deb, i-1) \;
      %\triSeg(tab, i+1, fin) \;
  %}
%\end{procedure}
% \begin{algorithmic}[1]
%   \Procedure{triSeg}{entier tab[], entier debut, entier fin}
%   \If{$debut<fin$}  // condition d'arr�t\pause
%     \State choisir �l�ment pivot $p$ entre debut et fin\pause
%     \State placer pivot en $i$, $\le$ pivot avant, $\geqslant$ pivot apr�s\pause
%     \State triSeg(tab, deb, i-1)
%     \State triSeg(tab, i+1, fin)
%   \EndIf
%   \EndProcedure
%%   \end{algorithmic}
%}
%}
%\only<5>{
  %{\small 
%\begin{procedure}[H]
  %\caption{triSeg(tableau\_entier tab, entier debut, entier fin)}
  %\If{$debut<fin$} {
    %\eIf{$fin - debut \leqslant 1$ \tcp{cas particulier : 2 �l�ments � trier}} {
      %\If{$tab[debut] > tab[fin]$} {
        %permuter tab[debut], tab[fin] \;
      %}
    %}
    %{
      %choisir �l�ment pivot $p$ entre debut et fin \;
      %placer pivot en $i$, $\leqslant$ pivot avant, $\geqslant$ pivot apr�s \;
      %\triSeg(tab, deb, i-1) \;
      %\triSeg(tab, i+1, fin) \;
    %}
  %}
%\end{procedure}
%% \begin{algorithmic}[1]
%%   \Procedure{triSegIns}{entier tab[], entier debut, entier fin}
%%     \If{$debut<fin$} // condition d'arr�t
%%       \If{\textcolor{blue}{$fin - debut \le seuil$}} // cas particulier 
%%         \State \textcolor{blue}{triIns(tab, debut, fin)}
%%       \Else
%%         \State choisir �l�ment pivot $p$ entre debut et fin
%%         \State placer pivot en $i$, $\le$ pivot avant, $\geqslant$ pivot apr�s
%%         \State triSeg(tab, deb, i-1)
%%         \State triSeg(tab, i+1, fin)
%%       \EndIf
%%     \EndIf
%%   \EndProcedure
%%   \end{algorithmic}
%}
%}
%\end{frame}


\subsection{Notion de r�cursivit� terminale}


\begin{frame}
  \frametitle{R�cursivit� terminale}%}

  \begin{itemize}
  \item A chaque appel r�cursif : empilement de donn�es\pause
  \item[$\Rightarrow$] limite au nombre d'appels possibles\pause
  \item Solution : r�cursivit� terminale
  \end{itemize}

  \pause
  \begin{defi}[R�cursivit� terminale]
    Une fonction est dite r�cursive terminale s'il n'y a aucune op�ration sur
    ses appels r�cursifs.
  \end{defi}
  \pause
  \begin{exemple}[Fonction r�cursive non terminale]
    Factorielle :
    \begin{semiverbatim}
      return n*fact(n-1);
    \end{semiverbatim}
  \end{exemple}
\end{frame}

\begin{frame}
  \frametitle{�criture de fonction r�cursive terminale}%}
  \begin{itemize}
  \item Id�e g�n�rale : utiliser un accumulateur
  \item Param�tre qui contient le r�sultat interm�diaire
  \end{itemize}
  \begin{exemple}[Factorielle r�cursive terminale]
    \begin{semiverbatim}
      def fact(int n, int acc):

      \hspace{1cm}if n==0 :

      \hspace{2cm}return acc

      \hspace{1cm}else:

      \hspace{2cm}return fact(n-1, n*acc)

    \end{semiverbatim}
  \end{exemple}
\end{frame}

\begin{frame}[fragile]
  \frametitle{Fonctionnement}%}
  Appel :
  \begin{semiverbatim}
    fact(4, 1)
  \end{semiverbatim}
  Si le langage sait optimiser la r�cursivit� terminale, pas d'empilement
  d'appels $\Rightarrow$ Modification des valeurs dans la pile.
  \vspace{2mm}

  \begin{tabular}{ll}
    \begin{minipage}{0.5\linewidth}
    \begin{semiverbatim}
      fact(\only<1>{4}\only<2>{3}\only<3>{2}\only<4>{1}\only<5->{0}, \only<1>{1}\only<2>{4}\only<3>{12}\only<4>{24}\only<5->{24});
    \end{semiverbatim}
  \end{minipage}&
  \begin{minipage}{0.5\linewidth}
    \begin{tabular}{l|r|}
      \hline
      acc & \only<1>{1}\only<2>{4}\only<3>{12}\only<4>{24}\only<5->{24}\\
       \hline
      n & \only<1>{4}\only<2>{3}\only<3>{2}\only<4>{1}\only<5->{0}\\
      \hline
      %0x1E3C85\\

    \end{tabular}
  \end{minipage}
\end{tabular}

\uncover<6>{
  \begin{alertblock}{Remarque}
    Python n'optimise pas la r�cursivit� terminale
  \end{alertblock}
}
\end{frame}

%\section{Eclipse}
%
%\begin{frame}
%\frametitle{Eclipse}
  %\begin{itemize}[<+->]
  %\item Disponible sur \url{http://www.eclipse.org}
  %\item Principe d'un IDE
  %\item Avantages / inconv�nients
    %\begin{itemize}
    %\item Debugger
    %\item Analyse syntaxique � la vol�e
    %\item Lourdeur
    %\end{itemize}
  %\item Installation / utilisation : voir didacticiels sur \emph{moodle}
  %\end{itemize}
%\end{frame}
%\section{Un mot sur le projet "LCLC"}
%\section{Horloge }

 
\begin{frame}	{\chrono}
	
\begin{center}
\colorbox{red}{\color{yellow} \hhmmss \factorclockfont{6.0} \tdtime}
\end{center}
\end{frame}

\end{document}


%%% Local Variables: 
%%% mode: latex
%%% TeX-master: t
%%% End: 

\section{Les tableaux}


\subsection{D�claration}



\begin{frame}
  \frametitle{Les tableaux}
  
  \begin{defi}[Tableau]
    Structure de donn�es contenant plusieurs �l�ments du m�me type
    (primitif ou composite).
  \end{defi}
\pause
  \begin{exemple}[D�claration de tableaux]
    \begin{semiverbatim}
      {\color[rgb]{0,0.7,0}{{\scriptsize // \textit{idEtudiants} est un tableau � une dimension de \emph{int}}}}
      
      int idEtudiants[]; 
		
			{\color[rgb]{0,0.7,0}{{\scriptsize//\textit{Notes} est un tableau � une dimension de \emph{char}}}}
			
      char[] notes;

      {\color[rgb]{0,0.7,0}{{\scriptsize //\textit{coordonnees} est un tableau � deux dimensions \emph{double}}}}
      
        double coordonnees[][];
    \end{semiverbatim}
  \end{exemple}
\end{frame}


\begin{frame}
  \frametitle{Cr�ation d'un tableau}
  \begin{itemize}
  \item N�cessit� de d�finir la taille du tableau
  \item[$\Rightarrow$] utilisation de \emph{new}
  \end{itemize}

\pause

  \begin{exemple}[Cr�ation de tableaux]
    \begin{semiverbatim}
      int idEtudiant[] = \textcolor{red}{new} int [20];

      char[] notes = \textcolor{red}{new} char [20];

      double coordonnees[][] = \textcolor{red}{new} double [10][5];
    \end{semiverbatim}
  \end{exemple}
  \pause
  \begin{exemple}[Cr�ation et initialisation de tableau]
    \begin{semiverbatim}
      char[] notes = \{'A','B','C','D','F'\};

      double coordonnees[][] = \{\{0.0,0.1\},\{0.2,0.3\}\};
    \end{semiverbatim}
  \end{exemple}
\end{frame}

\begin{frame}
  \frametitle{Cr�ation d'un tableau}
   \begin{exemple}[Cr�ation de tableaux]
    \begin{semiverbatim}
  
      \temporal<1> {\color{black}}{\color{black}}{\color{gray!25}} int ti[] = \textbf{new} int [10];
        
      \temporal<2> {\color{gray!25}}{\color{black}}{\color{gray!25}} char[] tc = \textbf{new} char [15];
         
      \temporal<3> {\color{gray!25}}{\color{black}}{\color{gray!25}} int t2[][] = \textbf{new} int[5][10];
          
      \temporal<4> {\color{gray!25}}{\color{black}}{\color{gray!25}}  int t21[][] = \{\{1, 2, 3\}, \{4, 5, 6\}\};
      
    \end{semiverbatim}
  \end{exemple}

 % \pdflatex{tableau1.png}
% \hspace{.4cm}
 %\vspace{-.1cm}

\only<1>{\includegraphics[width=4cm]{tableau1.jpg}}
\only<2>{\includegraphics[width=6cm]{tableau2.jpg}}
\only<3>{\includegraphics[width=6cm]{tableau3.jpg}}
\only<4>{\includegraphics[width=5cm]{tableau4.jpg}}



\end{frame}
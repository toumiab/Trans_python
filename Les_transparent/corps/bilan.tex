\section{Bilan}

\begin{frame}
  \frametitle{Progression}
  \begin{itemize}[<+-|alert@+>]
  \item Introduction, �l�ments de syntaxe
  \item Tris
  \item M�thodes
  \item R�cursivit�
  \item TAD
  \item Objet, h�ritage, polymorphisme
  \end{itemize}

  TD :
  \begin{itemize}
  \item 
  \only<1>{Hello World !}
  \only<2>{Tableaux, tris}
  \only<3>{M�thodes}
  \only<3>{\item Tableaux multidimentionnels, images}
  \only<4>{R�cursivit�}
  \only<5>{Pile d'objets}
  \only<6>{Insectes}
  \end{itemize}

  Th�me abord� :
  \begin{itemize}
  \item 
  \only<1>{Syntaxe Java, boucles, tests}
  \only<2>{Manipulation de tableaux, algorithmique}
  \only<3>{D�coupage de programme en sous-fonctions}
  \only<4>{Algorithmique, notion de pile d'appel, utilisation d'eclipse,
    utilisation du debugger}
  \only<5>{D�coupage d'un programme en types �l�mentaires, premier pas vers l'objet}
  \only<6>{D�coupage d'un programme en objets, h�ritage, polymorphisme}
  \end{itemize}
\end{frame}

\begin{frame}
  \frametitle{Progression, suite}
  \begin{itemize}[<+-|alert@+>]
  \item Fichiers
  \item Listes, Tables de hachage
  \item Arbres
  \item IHM
  \item JDBC
  \end{itemize}

  TD :
  \begin{itemize}
  \item 
  \only<1>{Fichiers}
  \only<2>{Cr�ation de liste, manipulation et utilisation de listes}
  \only<3>{Arbres binaires}
  \only<4>{Interface graphique}
  \only<5>{JDBC}
  \end{itemize}

  Th�me abord� :
  \begin{itemize}
  \item 
  \only<1>{Manipulation de fichiers textes}
  \only<2>{Cr�ation/manipulation de structure dynamique}
  \only<3>{Structures arborescentes, algorithmique}
  \only<4>{Couplage programme/interface}
  \only<5>{Utilisation de donn�es persistantes, passage relationnel/objet}
  \end{itemize}
\end{frame}

\begin{frame}
  \frametitle{Acquis}
  \begin{itemize}
  \item<1-> Cours : principes g�n�raux, pr�sentation et �tude d'algorithmes
  \item<2-> TD : mise en pratique sur des cas d'�cole
  \item<3-|alert@3> Projet : consolidation des acquis
  \end{itemize}

  \uncover<4->{Notions~:}
  \begin{itemize}
  \item<4-> Connaissances �l�mentaires d'algorithmique
  \item<5-> Connaissance d'un langage structur�
  \item<6-> Notions de conception objet
  \item<7-> Possibilit� d'utiliser la plupart des langages
  \item<8-> apr�s un temps d'adaptation...
  \end{itemize}
\end{frame}

\begin{frame}
  \frametitle{Suite ?}
  \begin{itemize}
  \item Fin de l'ann�e : stage de r�alisation\pause
  \item Branche hydrographie
    \begin{itemize}
    \item Utilisation de Python, Scilab, Matlab
    \end{itemize}\pause
  \item Branche �lectronique
    \begin{itemize}
    \item Langage C
    \item Syst�me
    \item Mod�lisation
    \end{itemize}\pause
  \item M�canique\pause
    \begin{itemize}
    \item ???
    \end{itemize}\pause
  \item Modules
    \begin{itemize}
    \item Introduction � la recherche op�rationnelle\pause
    \item Prototypage et g�nie logiciel
    \end{itemize}
  \end{itemize}
\end{frame}

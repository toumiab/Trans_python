
\section{Typages des variables }

\subsection{Les types de base}


\begin{frame}
\frametitle{Typage des variables}
\only<1-2>{
\begin{itemize}
\item<1-2> Python: langage  typ� dynamiquement 
\item<2> [$\Rightarrow$] typage lors de l'affectation
\item<2> Une variable peut changer de type au cours de l'ex�cution
\end{itemize}
 \pause 
	\begin{remarque}
    \begin{itemize}
			\item Pour conna�tre le type d'une variable ou expression, on peut utiliser la fonction \emph{type()}
			\item La fonction \emph{print()} permet d'afficher la valeur d'une variable
		\end{itemize}
		
  \end{remarque}

}
\only<3->{
\begin{itemize}
\item<3-> \textbf{Les types int�gr�s de Python}
\begin{itemize}
    \item<4-> Entiers sign�s (compl�ment � deux) : le type \emph{int} 
    \item<4-> R�els (IEEE 754) : le type \emph{float} en double pr�cision (64 bits)
    \item<4-> Complexes : le type \emph{complex}
		\item<5-> Bool�ens : le type \emph{bool}
		\item<6-> Cha�nes de caract�res : type \emph{str}
		\item<7-> Listes : type \emph{list}
		\item<8-> Dictionnaires : type \emph{dict}
		\item<9-> Tuples : type \emph{tuple}
		\item<10-> Ensembles modifiables et immuables : le type \emph{set} et le type \emph{frozenset}
    \end{itemize}
\end{itemize}
}
\end{frame}






%\section{Typages des variables }

\subsection{Les types de base}


%\begin{frame}
%\frametitle{Typage des variables\chrono}
%\only<1-2>{
%\begin{itemize}
%\item<1-2> Python: langage  typ� dynamiquement 
%\item<2> [$\Rightarrow$] typage lors de l'affectation
%\item<2> Une variable peut changer de type au cours de l'ex�cution
%\end{itemize}
 %\pause 
	%\begin{remarque}
    %\begin{itemize}
			%\item Pour conna�tre le type d'une variable ou expression, on peut utiliser la fonction \emph{type()}
			%\item La fonction \emph{print()} permet d'afficher la valeur d'une variable
		%\end{itemize}
		%
  %\end{remarque}
%
%}
%\only<1->{
%\begin{itemize}
%\item<1-> \textbf{Les types int�gr�s de Python}
%\begin{itemize}
    %\item<2-> Entiers sign�s (compl�ment � deux) : le type \emph{int} 
    %\item<2-> R�els (IEEE 754) : le type \emph{float} en double pr�cision (64 bits)
    %\item<2-> Complexes : le type \emph{complex}
		%\item<3-> Bool�ens : le type \emph{bool}
		%\item<4-> Cha�nes de caract�res : type \emph{str}
		%\item<5-> Listes : type \emph{list}
		%\item<6-> Dictionnaires : type \emph{dict}
		%\item<7-> Tuples : type \emph{tuple}
		%\item<8-> Ensembles modifiables et immuables : le type \emph{set} et le type \emph{frozenset}
    %\end{itemize}
%\end{itemize}
%}
%\end{frame}

\begin{frame}
\frametitle{Typage des variables}%\chrono} 
\begin{itemize}
\item \textbf{Les types int�gr�s de Python}
\begin{itemize}
\item Les types simples : 
	\begin{itemize}
		\item Entiers sign�s (\emph{int}), R�els (IEEE 754) (\emph{float}) et  Complexes (\emph{complex}), Bool�ens (\emph{bool})
	\end{itemize}
\item Les types composites (containers): 
		\begin{itemize}
		  \item \textbf{Les s�quences} : Cha�nes de caract�res (\emph{str}) ;		Listes (\emph{list}) et  Tuples (\emph{tuple})
			\item \textbf{Les maps }(hashs) : Dictionnaires (\emph{dict})
			\item \textbf{Les ensembles} : le type \emph{set} et le type \emph{frozenset}
		\end{itemize}
 \end{itemize}
\end{itemize}


\end{frame}
\begin{frame}
\frametitle{Types des variables}%\chrono}
  \begin{exemple}
    \begin{semiverbatim}
 >>> i = 42 

 >>> type(i)

 <class 'int'>

 >>> i = 'indice' 

 >>> type(i)

 <class 'str'>

 >>> i = 42.0 

 >>> type(i)

 <class 'float'>

 >>> print(i)

42.5

>>>
    \end{semiverbatim}
  \end{exemple}
\end{frame}




%% teste si compil avec pdflatex ou avec latex
\input ifpdf.sty

\ifpdf
\documentclass[pdftex,beamer]{beamer}
\else
\documentclass[xcolor=pst,dvips,beamer]{beamer}
\fi

%\documentclass[handout]{beamer}


% Try the class options [notes], [notes=only], [trans], [handout],
% [red], [compress], [draft], [class=article] and see what happens!

% Copyright 2003 by Till Tantau <tantau@users.sourceforge.net>.
%
% This program can be redistributed and/or modified under the terms
% of the LaTeX Project Public License Distributed from CTAN
% archives in directory macros/latex/base/lppl.txt.

%\usepackage{pstricks} % To use the standard "xcolor" package with PSTricks
%\usepackage{pst-node}
%\usepackage{pst-plot}
%\usepackage{pst-tree}

\usepackage{pgf,pgfarrows,pgfnodes,pgfautomata,pgfheaps,pgfshade}
\usepackage{amsmath,amssymb,euscript}
\usepackage[latin1]{inputenc}
\usepackage{colortbl}
\usepackage[frenchb]{babel}
%\usepackage{gastex}

%\usepackage{times}
%\usepackage{lmodern}
%\usepackage[T1]{fontenc} 

\usepackage{multicol}

%\ifpdf
%\usepackage[pdftex]{graphicx}
%\DeclareGraphicsExtensions{.pdf,.png,.mps}
%\else
%\usepackage[dvips]{graphicx}
%%\DeclareGraphicsExtensions{.eps}
%\fi


%% hyperref doit etre charge en dernier
%\ifpdf
%\usepackage[pdftex,colorlinks]{hyperref}
%\else
%\usepackage[dvips]{hyperref}
%\fi
%%


\unitlength 1cm
\def\nbR{\ensuremath{\mathrm{I\!R}}} % IR
\def\nbN{\ensuremath{\mathrm{I\!N}}} % IN
\def\Coup{\ensuremath{\mathcal C}} % IN
\def\Cont{\ensuremath{\EuScript C}}
\def\Dom{\ensuremath{\EuScript D}}

\newcommand{\NoeudLst}[2] {\begin{tabular}{|l|l|} \hline #1&#2\\ \hline \end{tabular}}

% \nbQ   : Nombres Rationnels Q
\def\nbQ{{\mathchoice {\setbox0=\hbox{$\displaystyle\rm
Q$}\hbox{\raise
0.15\ht0\hbox to0pt{\kern0.4\wd0\vrule height0.8\ht0\hss}\box0}}
{\setbox0=\hbox{$\textstyle\rm Q$}\hbox{\raise
0.15\ht0\hbox to0pt{\kern0.4\wd0\vrule height0.8\ht0\hss}\box0}}
{\setbox0=\hbox{$\scriptstyle\rm Q$}\hbox{\raise
0.15\ht0\hbox to0pt{\kern0.4\wd0\vrule height0.7\ht0\hss}\box0}}
{\setbox0=\hbox{$\scriptscriptstyle\rm Q$}\hbox{\raise
0.15\ht0\hbox to0pt{\kern0.4\wd0\vrule height0.7\ht0\hss}\box0}}}}
%

\newenvironment{algorithme}[2]{
  \begin{figure}[htbp]
    \hrule \def\tempa{#1} \def\tempb{#2}
    \begin{center}
      \begin{tabbing}}
      {\end{tabbing}
    \end{center}
    \hrule
    %\caption{\tempa}
    %\label{\tempb}
  \end{figure}}
 \usepackage{lmodern}
%% Commandes � utiliser avec l'environnement algorithme :
\providecommand{\algemph}{\textbf}
\providecommand{\algverb}{\texttt}
\providecommand{\proc}{\algemph{Proc�dure }}
\providecommand{\func}{\algemph{Fonction }}
\providecommand{\vars}{\algemph{variables : }}
\providecommand{\precond}{$\triangleright$ \algemph{Pr�conditions : }}
\providecommand{\debut}{\algemph{d�but }}
\providecommand{\fin}{\algemph{fin}}
\providecommand{\faire}{\algemph{faire }} 
\providecommand{\pour}{\algemph{Pour }} 
\providecommand{\tantque}{\algemph{Tant que }}
\providecommand{\pourchaque}{\algemph{Pour chaque }}
\providecommand{\repeter}{\algemph{R�p�ter }}
\providecommand{\jusqua}{\algemph{jusqu'� }}
\providecommand{\si}{\algemph{Si }} 
\providecommand{\alors}{\algemph{alors }}
\providecommand{\sinon}{\algemph{sinon }}
\providecommand{\aff}{\leftarrow} % en mode math ...

\newcommand{\macoulr}[1]{\color{red}#1\color{black}}
\newcommand{\macoulb}[1]{\color{blue}#1\color{black}}
% Use some nice templates
\beamertemplatetransparentcovereddynamic

\theoremstyle{definition}
\newtheorem{defi}{D�finition}
\theoremstyle{example}
\newtheorem{remarque}{Remarque}
\newtheorem{remarques}{Remarques}
\theoremstyle{example}
\newtheorem{exemple}{Exemple}
\theoremstyle{theorem}
\newtheorem{theoreme}{Th�or�me}
\theoremstyle{theorem}
\newtheorem{propriete}{Propri�t�s}
\theoremstyle{theorem}
\newtheorem{lemme}{Lemme}
\theoremstyle{theorem}
\newtheorem{corollaire}{Corollaire}

\title[Java]{Langage et algorithmique}
\author[RM]{Rod�ric Moiti�}
%\date[DLT 2003]{Developments in Language Theory Conference, 2003}
\date[2007]{\today}

\usetheme{Ensieta2}

\mode<handout>{
  \usecolortheme{dove}
}


\begin{document}

\frame{\titlepage}

%\begin{frame}
%  \frametitle{Sommaire}
%  \tableofcontents[hideallsubsections]
%\end{frame}

\AtBeginSection[]
{
  \begin{frame}<beamer>
    \frametitle{Sommaire}
    \tableofcontents[current,hideallsubsections]
  \end{frame}
}

\AtBeginSubsection[]
{
  \begin{frame}<beamer>
    \frametitle{Sommaire}
    \tableofcontents[sectionstyle=show/shaded,subsectionstyle=show/shaded/hide]
  \end{frame}
}


%%%
\section{D�roulement du cours }
\begin{frame}
  \frametitle{D�roulement\chrono}
	\initclock  % This must be used one time to initialize the clock
  \begin{itemize}
  \item Premi�re partie  : les bases de Python
  \item Deuxi�me partie :  objet et suite de l'algorithmique 
  \item Projet informatique : deux parties (sans et avec une interface graphique)
  \end{itemize}
  \pause
  \begin{itemize}
  \item Programmation imp�rative et objet
  \item Notions d'architecture logicielle
  \end{itemize}
  \pause
	\begin{itemize}
  \item Le cours est d�coup� :  CM ($\cong$30 min) + Pause (7 min) + CM 
  \item Utilisation des cartons (\textcolor{green}{vert}/\textcolor{red}{rouge}) pour les exercices Q/R
  \end{itemize}
  Plateforme : \url{https://moodle.ensta-bretagne.fr}
\end{frame}
%\date{\tddate\ \ \tdtime}
%\begin{frame}	
%
  %\frametitle{horloge \chrono}
	%
%\begin{center}
%\colorbox{red}{\color{yellow} \hhmm \factorclockfont{6.0} \tdtime}
%\end{center}
%\end{frame}

%%% Local Variables: 
%%% mode: latex
%%% TeX-master: 
%%% End: 

\section{Introduction}


%\subsection{Histoire de Python}

%http://histoire.info.online.fr

%\begin{frame}
%\frametitle{Pr�histoire}
%\only<-2,4,8>{
  %\begin{itemize}
  %\item<1-> -500 : Moyen Orient $\Rightarrow$ premier ``outil'' de calcul :
    %l'abaque et le boulier.
  %\item<2-> 1623 : Wilhelm Schickard $\Rightarrow$ ``horloge calculante''. 
    %Calcul m�canique (roues dent�es) : additions, soustractions,
    %multiplications et m�morisation r�sultats interm�diaires.
  %\item<4-> 1642 : Pascal $\Rightarrow$ Pascaline. Additions et soustractions. 
    %Premi�re machine � calculer de l'histoire
%%  \item 1728 : Falcon  $\Rightarrow$ m�tier � tisser utilisant cartes perfor�es.
%%  \item 1770 : Hahn (Allemagne) $\Rightarrow$ premi�re machine � calculer
%%    4 op�rations (cylindre dent� invent� par
%%    Leibnitz en 1671).
  %\item<8-> 1833 : Babbage  $\Rightarrow$ machine � diff�rences
    %puis machine analytique $\simeq$ ordinateur moderne : unit� de
    %calcul, m�moire, registre et entr�e des donn�es (carte perfor�e).
    %Jamais termin�
  %\end{itemize}
%}
  %\only<3>{
    %\begin{center}
      %\includegraphics[height=6cm]{images/CalculatingClock}
    %\end{center}
  %}
  %\only<5>{
    %\begin{center}
      %\includegraphics[height=6cm]{images/Pascaline}
    %\end{center}
  %}
  %\only<6>{
    %\begin{center}
      %\includegraphics[height=5cm]{images/Pascaline2}
    %\end{center}
  %}
  %\only<7>{
    %\begin{center}
      %\includegraphics[height=6cm]{images/Pascaline3}
    %\end{center}
  %}
  %\only<9>{
    %\begin{center}
      %\includegraphics[height=6cm]{images/DifferenceEngine}
    %\end{center}
  %}
%
%\end{frame}
%
%%
%%Ada was 18 years old when she met Charles Babbage. Knowing of her interest in
%%mathematics and machines, Babbage invited her to study the plans for his
%%engine. She translated the paper written by Menabrea, and added her own
%%thoughts and footnotes. When she was done, the paper was three times longer
%%and it was published by Babbage with only the initials "A.A.L." to show who
%%had authored it. It wasn't that Babbage kept Ada's name off the paper, it was
%%that the customs of the day did not allow Ada, a woman and a wife of an earl
%%publish books: it just wasn't ladylike. 
%
%%Years later it was discovered that "A.A.L." was, indeed, Lady Ada
%%Lovelace. It was also discovered that the paper she wrote was really the
%%first explanation of how a a set of instructions for computing machine might
%%be programmed. Ada was the world's first computer programmer. 
%%
%
%\begin{frame}
%\frametitle{Pr�histoire}
  %\begin{itemize}[<+->]
  %\item 1840 : Ada Lovelace (collaboratrice Babbage) : principe des it�rations
    %successives dans l'ex�cution d'une 
    %op�ration. Nomm� selon le math�maticien arabe \emph{Al Khawarizmi (780-850)}~:
    %algorithme. 
  %\item 1854 : Boole $\Rightarrow$ tout
    %processus logique peut �tre d�compos� en une suite d'op�rations logiques
    %(ET, OU, NON) et deux �tats (ZERO-UN, OUI-NON, VRAI-FAUX).
%%  \item 1884 : Herman Hollerith $\Rightarrow$ tabulatrice � cartes perfor�es
%%    Premi�re machine � traiter l'information.
%%  \item 1886 : Don E. Felt de Chicago : Comptometer. Calculatrice � touches.
%%    1889 : premi�re calculatrice de bureau avec imprimante.
  %\item 1904 : Premier tube � vide (diode) par John Fleming.
  %\end{itemize}
%\end{frame}
%
%\begin{frame}
%\frametitle{Pr�histoire}
%\only<1-4>{
  %\begin{itemize}
  %\item<1-> 1935 : IBM 601 : calculateur � relais utilisant
    %des cartes perfor�es. 1 multiplication/s. Vendu � 1500 exemplaires.
  %\item<2-> 1937 : Alan M. Turing document sur les nombres
    %calculables : utilisation de Machine de Turing $\Rightarrow$
    %ind�cidabilit� de HALT.
%%  \item 1938 : Th�se de Shannon : parall�le entre
%%    circuits �lectriques et alg�bre Bool�enne. Chiffre binaire~: bit.
  %\item<3-> 1939 : John Atanasoff et Clifford Berry : additionneur 16
    %bits binaire. Premier calculateur � tubes � vide.
  %\item<4-> 1940 : D�chiffrage des messages de l'arm�e Allemande : les Anglais
    %cr�ent les calculateurs Robinson  et Colossus (direction de Turing). 
    %Premi�res machines utilisant arithm�tique binaire,
    %horloge interne, m�moire tampon, lecteurs de bande, op�rateurs
    %bool�ens, sous programmes et imprimantes. "Secret d�fense" jusqu'en 1975.
%
  %\end{itemize}
%}
  %\only<5>{
    %\begin{center}
      %\includegraphics[height=6cm]{Colossus1}
    %\end{center}
  %}
  %\only<6>{
    %\begin{center}
      %\includegraphics[height=6cm]{Colossus2}
    %\end{center}
  %}
%
%\end{frame}
%
%\begin{frame}
%\frametitle{Pr�histoire}
%\only<1,3,5>{
  %\begin{itemize}
  %\item<1-> 1941 : Calculateur binaire ABC par John Atanasoff et
    %Clifford Berry. Machine � lampes, m�moire et
    %circuits logiques. 
    %Premier calculateur � alg�bre de Boole. M�moire (2 tambours) : 60 mots de
    %50 bits. Cadence 60 Hz, 1 addition/s. 
    %Premier vrai ordinateur (?) programme non stock� en m�moire.
  %\item<3-> 1941 : Konrad Zuse : Z3, Premier calculateur avec programme
    %enregistr�. 
    %Machine de 2600 relais, console pour l'op�rateur et
    %lecteur de bandes contenant (instructions). M�moire : 64 nombres de 22
    %bits. 
    %4 additions/s et 1 multiplication en 4 secondes. D�truite dans un
    %bombardement alli� en Avril 1945.
  %\item<5-> 1945 : Un insecte coinc� dans les circuits bloque le 
    %calculateur Mark I. La math�maticienne Grace Murray Hopper d�cide 
    %que tout ce qui arr�te un programme s'appellera \textsc{bug}. 
  %\end{itemize}
%}
  %\only<2>{
    %\begin{center}
      %\includegraphics[height=6cm]{ABC}
    %\end{center}
  %}
  %\only<4>{
    %\begin{center}
      %\includegraphics[height=6cm]{Z3_replic}
    %\end{center}
  %}
%
%\end{frame}
%
%
%\begin{frame}
  %\frametitle{Premiers ordinateurs}
%
  %\only<1,5->{
    %\begin{itemize}
    %\item<1-> 1946 : ENIAC (Electronic Numerical Integrator and
      %Computer) par P. Eckert et J. Mauchly. Programmation par c�blage des
      %diff�rents �l�ments. 
      %19000 tubes, 30 tonnes,  72 m2, consomme 140 kilowatts. 
      %Horloge : 100 KHz. Vitesse : $\simeq$ 330 multiplications/s
    %\item<5-> D�cembre 1947 : Invention du transistor par William Bradford
      %Shockley, Walter H. Brattain et John Bardeen dans les laboratoires de
      %Bell Telephone.
    %\item<6-> Juin 1948 : NewMan, Williams (universit� de
      %Manchester) terminent un prototype appel� Manchester Mark I
      %avec m�moire compos�e de tubes cathodiques.
      %Stocker 1 bit : allumer un point sur le tube. Pour lire : pointer
      %le rayon au m�me endroit et mesurer la tension. 
      %M�moire de 1024 bits sur un tube. 
      %Machine programm�e en binaire.
    %\end{itemize}
  %}
  %\only<2>{
    %\begin{center}
      %\includegraphics[height=7cm]{eniac4}
    %\end{center}
  %}
  %\only<3>{
    %\begin{center}
      %\includegraphics[height=7cm]{eniac5}
    %\end{center}
  %}
  %\only<4>{
    %\begin{center}
      %\includegraphics[height=7cm]{eniac6}
    %\end{center}
  %}
%\end{frame}
%
%
%\begin{frame}
  %\frametitle{Premiers ordinateurs}
  %\only<1,7->{
    %\begin{itemize}
    %\item<1,7-> 1949 - 1951 : Premier ordinateur temps r�el : le Whirlwind cr�e
      %au MIT par Jay Forrester, Ken Olsen et leur �quipe.
    %\item<7-> Le Whirlwind : 
      %\begin{itemize}
      %\item 27 instructions
      %\item 20000 instructions/s
      %\item M�moire : 1024 registres de 16 bits
      %\item Entr�e sortie de donn�es par clavier / ruban perfor�
      %\item Affichage de graphiques sur �cran cathodique en 32x32
      %\item Surface occup�e : 300 $m^2$ - Consommation : 150 kW
      %\end{itemize}
    %\item<8-> 1950 : Invention de l'assembleur par Maurice V. Wilkes de
      %l'universit� de Cambridge.
    %\item<9-> 1951 : Invention du premier compilateur A0 par Grace Murray
      %Hopper. Code source $\leadsto$ binaire
    %\end{itemize}
  %}
  %\only<2>{
    %\begin{center}
      %\includegraphics[width=8cm]{whirlwind}\\
      %Salle des machines du Whirlwind
    %\end{center}
  %}
%
  %\only<3>{
    %\begin{center}
      %\includegraphics[height=6cm]{whirlwind2}\\
      %Une all�e du Whirlwind
    %\end{center}
  %}
  %\only<4>{
    %\begin{center}
      %\includegraphics[height=6cm]{detail-whirlwind}\\
      %Multiplicateur 5 bits
    %\end{center}
  %}
  %\only<5>{
    %\begin{center}
      %\includegraphics[height=6cm]{ecran-whirlwind}\\
      %�cran du Whirlwind
    %\end{center}
  %}
  %\only<6>{
    %\begin{center}
      %\includegraphics[height=6cm]{tubes-whirl}\\
      %M�moire � tubes : 64 mots de 16 bits par tube
    %\end{center}
  %}
%
%\end{frame}
%
%
%\begin{frame}
  %\frametitle{Premiers ordinateurs}
%
%\only<1,3,5->{
  %\begin{itemize}
  %\item<1,3,5-> 1955 : IBM 704. 
    %Premi�re machine avec coprocesseur math�matique. 
    %Puissance : 5 kFLOPS.
    %M�moire � tores de ferrite de 32768 mots de 36 bits.
    %Machine tr�s fiable : une panne par semaine.
    %D�veloppement de FORTRAN.
  %\item<1,3,5-> 1956 : Premier ordinateur � transistors : le TRADIC
  %\item<1,3,5-> 1956 : Premier disque dur (IBM), le RAMAC 305 (Random Access Method of
    %Accounting and Control).
    %50 disques de 61 cm de diam�tre (5 Mo).
  %\item<3,5-> 1958 : Ordinateur � transistors : le CDC 1604.
  %\item<5-> 1968 : Premiers ordinateurs � circuits int�gr�s
  %\item<5-> 1970 : Premi�re puce m�moire (Intel) carr� de 0.5 mm de c�t�
    %(capacit� : 1kBit soit 128 octets) 
  %\end{itemize}
%}
  %\only<2>{
    %\begin{center}
      %\includegraphics[height=6cm]{ramac305}
    %\end{center}
  %}
  %\only<4>{
    %\begin{center}
      %\includegraphics[width=11cm]{cdc1g}
    %\end{center}
  %}
%
%
%\end{frame}
%
%\begin{frame}
  %\frametitle{Les temps modernes}
%
  %\only<1>{
  %\begin{itemize}
  %\item Novembre 1971 : Intel met en vente le premier microprocesseur
    %\begin{itemize}
    %\item Processeur 4 bits  108 KHz
    %\item 640 octets de m�moire
    %\item 60000 instructions/s
    %\item 2300 transistors en 10$\mu$
    %\item Prix : \$200 
    %\end{itemize}
  %\item Avril 1972 : Premier microprocesseur 8 bits : le 8008 (Intel)
  %\item 1974 : Le 68000 (Motorola)
  %\item 1974 : Le 1802 tournant � 6.4 MHz (RCS) : premier RISC
  %\item Juin 1976 : microprocesseur 16 bits : le TMS 9900 (TI)
  %\item 1976 : le CRAY I
  %\end{itemize}
%}
  %\only<2>{
    %\begin{center}
      %\includegraphics[height=6cm]{Cray-1}
    %\end{center}
  %}
  %\only<3>{
    %\begin{center}
      %\includegraphics[height=6cm]{Cray-1-inside}
    %\end{center}
  %}
%\end{frame}

\subsection{Les langages}


\begin{frame}
  \frametitle{Histoire des langages}
  
  \begin{itemize}[<+->]
  \item D�but : langage machine
  \item Assembleur en 1950
  \item Fortran en 1954
	\item Matlab 1970 
  \item C en 1971
	\item Python en 1994 (premi�re id�e en 1990)
	\item Java  en 1995
	
  \end{itemize}
  \pause
  Historique : \href{file:lang_a4.pdf}{http://www.levenez.com}

  \pause
  Langage choisi~: Python
\end{frame}

\begin{frame}
  \frametitle{Histoire des langages}
 \only<1->{
 Historiques des versions \\\url{https://en.wikipedia.org/wiki/History_of_Python}
  \begin{itemize}
     \item  Python 1.0 - January 1994
			\only<2>{
			\begin{itemize} 
					\item Python 1.5 - December 31, 1997
					\item Python 1.6 - September 5, 2000
			\end{itemize}
			}
			
  \item <3-> Python 2.0 - October 16, 2000
	\only<4>{
		\begin{itemize}
		 \item Python 2.1 - April 17, 2001
     \item Python 2.2 - December 21, 2001
     \item Python 2.3 - July 29, 2003
     \item Python 2.4 - November 30, 2004
     \item Python 2.5 - September 19, 2006
     \item Python 2.6 - October 1, 2008
     \item Python 2.7 - July 3, 2010
	 \end{itemize}
	}
  \item<5-> Python 3.0 - December 3, 2008
	 \only<6>{
	\begin{itemize} 
		\item  Python 3.1 - June 27, 2009
		\item  Python 3.2 - February 20, 2011
		\item  Python 3.3 - September 29, 2012
		\item  Python 3.4 - March 16, 2014
  \end{itemize}
	}
\end{itemize}
}
\end{frame}

\subsection{Langage Python}

\begin{frame}
  \frametitle{Langage Python}
  \only<10>{
    \centering
		\includegraphics[width=10cm]{images/comparaisonLangages.png}
		}
	  \only<11>{
		\includegraphics[width=10cm]{images/interpreter.png}
	
	}
	\only<1-9>{
    \begin{itemize}[<+->]
    \item  Python est interpr�t� 
    \item  Python est orient� objet et permet une programmation imp�rative
    \item  Python est fortement typ�
		\item  Python est typ� dynamiquement
    \item  Python assure la gestion de la m�moire
    \item  Python est multit�che
		\item  Python interagit avec les programmes des autres langages
		\item  Python est ind�pendant de toute plate-forme
		\item  Python est agr�able et facile � lire
    \end{itemize}
  }

\end{frame}

\begin{frame}
  \frametitle{Ex�cution d'un programme Python}
	\only<1>{
    \begin{remarque}[Un programme Python]
		Un programme est une suite d'instructions Python qui peut �tre ex�cut� de diff�rentes  mani�res :
		\begin{itemize}
			\item \textbf{En mode interactif} 
			\item \textbf{En script } 
		\end{itemize}
		\end{remarque}
		}
	\only<2-5>{
	\begin{itemize}
		\item \textbf{En mode interactif} :
		  
		\begin{itemize}
				\item avec l'\emph{interpr�teur python} : via un terminal de commandes lanc� avec la commande 'python' ou via le \emph{shell python}
				 \begin{itemize}
					\item  Les instructions python � ins�rer apr�s l'invite de commande ($>>>$), ou � l'invite de poursuit (...)  
					\item  Les instructions sont ex�cut�es imm�diatement
					\item  Le r�sultat d'une instruction est affich� sur le terminal		
   		   \end{itemize}
				 \only<2>{
				 \includegraphics[width=6cm]{images/console.png} 
				 }
				\item <3-> avec \emph{ipython} : est une sourcouche de l'interpr�teur Python offrant une plusieurs facilit�s. En version web : \emph{IPython Notebook }
				 \only<3>{
				 \includegraphics[width=6cm]{images/ipython.png} 
				 }
				 \only<4>{
				 \includegraphics[width=6cm]{images/ipython1.png} 
				 }
					\item <5->  avec \emph{bpython} : analogue � ipython en plus l�ger et plus commode, il propose une coloration syntaxique sur la ligne de commande
			     	 \only<5>{
				 \includegraphics[width=7cm]{images/bpython.png} 
				 }
		\end{itemize}
	\end{itemize}
	}			
\end{frame}






\begin{frame}
   \frametitle{Ex�cution d'un programme Python}
  %\only<2>{
    %\includegraphics[width=11cm]{images/compil1}
  %}
  \only<2>{
    \includegraphics[width=11cm]{images/compil2}
  }
  \only<4>{
    \includegraphics[width=11cm]{images/compil3}
  }
  \only<1,3>{
    \begin{itemize}
		\item <1,3> \textbf{En mode script} (Programming mode): 
      \begin{itemize}
			  \item<1>�criture du programme Python (�diteur de texte) dans un fichier portant l'extension \emph{.py}
				\item<3> Ex�cution (\$ \texttt{python monprogramme.py})
      \end{itemize}
    \end{itemize}
  }
\end{frame}

\subsection{Impl�mentations de Python}
\begin{frame}
  \frametitle{Impl�mentations de Python}

  \begin{itemize}
  \item Autres impl�mentations Python : \url{https://www.python.org/download/alternatives/}
    \begin{itemize}
    \item Python (Cpython), cod�e en C
		\item IronPython, cod�e en C$\#$ et fonctionne sur les plate-formes .NET
    \item Jython, Python s'�x�cutant sur une VM java
    \item PyPy, Python plus rapide, cod� avec le compilateur JIT)
    \item Stackless Python, une branche de CPython supportant les microthreads
    \end{itemize} 
	\end{itemize}
\end{frame}


\subsection{Installation et environnement de d�veloppement}

\begin{frame}
 \frametitle{Installation d'un environnement Python}
  
\begin{itemize}
   \item Sous Windows, Python n'est pas install� par d�faut. \pause
	 \item Sous GNU/Linux et Mac OS X, Python et la librairie standard sont int�gr�s \pause
	 \begin{itemize}
				 \item depuis Ubuntu 14.04 on dispose en parall�le de Pyhon v2 et v3
				 \item Mac OS X 10.8, 10.9 et 10.10, Apple propose que Python 2.7
				\pause
		\end{itemize}
	  \item Installation :
	 	 \begin{itemize}
		 \item Windows : \emph{WinPython} distribution sp�cifique � Windows et tr�s facile � mettre en \oe uvre. Il int�gre notamment : IPython, Spyder, NumPy, SciPy, MatplotLib, Pandas, SymPy, PIP... \pause
		\item GNU/Linux Ubuntu $\geq$ 14.04 : installation standard via les d�p�ts officiels Ubuntu de Canonical
		\item Mac OS X : \emph{Anaconda } distribution Python multiplateforme tr�s r�pandue dans les milieux scientifiques
	 \end{itemize}
	\end{itemize}
	
\end{frame}	
	
\begin{frame}
 \frametitle{Environnement de d�veloppement}
  
\begin{itemize}
   \item bloc-note, notepad++, vim, emacs ;
   \item IDE (Integrated Development Environment) : 
	 \url{https://wiki.python.org/moin/IntegratedDevelopmentEnvironments}
    \begin{itemize}
			\item Netbeans (Sun) avec Python/jPython
			\item Eclipse (IBM) avec le plugin PyDev
			\item \emph{Spyder }
			\item Komodo IDE
			\item LiCipse avec PyDev
			\item pyCharm
			\item ....
   \end{itemize}
		\end{itemize}
		
\end{frame}	

\begin{frame}
\frametitle{Python et le calcul scientifique}
	\begin{itemize}
	%	\justifying
		\item rapide � apprendre (mais long � ma�triser) ;
		\item alternative (tr�s) s�rieuse � Matlab, Scilab, Octave ;
		\item biblioth�ques de calcul et de visualisation tr�s compl�te et performante ;
		\item parall�lisation possible facilement (efficacit�, etc.) ;
		\item communaut� nombreuse et active ;
		\item multi-plateformes (scripts python sur Abaqus sur vos machines -- Windows -- et sur le cluster -- Linux)
		
	\end{itemize}
\end{frame}

% ---------------------------------------------------------------------------------Slide ----
\begin{frame}
\frametitle{Python et le calcul scientifique}
	\begin{itemize}
		%\justifying
		\item rapide � apprendre (mais long � ma�triser) ;
		\item alternative (tr�s) s�rieuse � Matlab, Scilab, Octave ;
		\item biblioth�ques de calcul et de visualisation tr�s compl�te et performante ;
		\item parall�lisation possible facilement (efficacit�, etc.) ;
		\item communaut� nombreuse et active ;
		\item multi-plateformes (scripts python sur Abaqus sur vos machines -- Windows -- et sur le cluster -- Linux)
		
	\end{itemize}
\end{frame}


% ---------------------------------------------------------------------------------Slide ----
\begin{frame}
\frametitle{Les principales librairies}
\textbf{1. Calcul scientifique} \\
\footnotesize{Pour plus d'information, consulter~: \url{http://numpy.org/} et \url{http://scipy.org/}}

	\begin{columns}
	    % colonne 1
		\begin{column}{5cm}
			\begin{center}
				numpy \\
				\includegraphics[width=1.5cm]{./images/numpylogo.png}
			\end{center}
		
				\begin{itemize}
		\item broadcasting
		\item multiplication de matrice
		\item traitement du signal
		\item traitement d'images	
		\end{itemize}	
		
			
		\end{column}
		% colonne 2
		\begin{column}{6.5cm}
			\begin{center}
				scipy \\
				\includegraphics[width=1.5cm]{./images/scipylogo.png}
			\end{center}
			\begin{itemize}	
     \item optimisation 
		\item interpolation
		\item int�gration num�rique
		\item alg�bre lin�aire 
	\end{itemize}
			
			
		\end{column}
	    % colonne 3
		%\begin{column}{0.33\textwidth}
			%\begin{center}
				%matplotlib \\
				%\includegraphics[width=2cm]{./figs/matplotliblogo.png}
			%\end{center}
		%\end{column}	
	\end{columns}
	
\end{frame}



% ---------------------------------------------------------------------------------Slide ----
\begin{frame}
\frametitle{Les principales librairies}
\textbf{2. Visualisation de donn�es}\\
\footnotesize{Plus d'informations sur les sites des modules~: \verb!http://matplotlib.org/!.}
	\begin{center}
		matplotlib \\
		%\includegraphics[width=2cm]{./figs/matplotliblogo.png}
	\end{center}
	\vspace{-0.5cm}
	\begin{columns}
	    % colonne 1
		\begin{column}{0.5\textwidth}
			\begin{center}
				\includegraphics[width=3cm]{./images/errorbar_limits_01.png} \\
				\includegraphics[width=3.5cm]{./images/streamplot_demo_features_00.png}
			\end{center}
		\end{column}
		% colonne 2
		\begin{column}{0.5\textwidth}
			\begin{center}
				\includegraphics[width=3cm]{./images/griddata_demo.png} \\
				\includegraphics[width=3cm]{./images/trisurf3d_demo.png}
			\end{center}
		\end{column}
	\end{columns}
	
\end{frame}


% ---------------------------------------------------------------------------------Slide ----
\begin{frame}{Les autres librairies}{}
	\begin{itemize}
		%\justifying
		\item numexpr (\verb!https://code.google.com/p/numexpr/!) ;
		\item NLopt (\verb!http://ab-initio.mit.edu/wiki/index.php/NLopt!) ;
		\item scikits (\verb!https://scikits.appspot.com/!) ;
		\item pyQt4 (\verb!http://www.riverbankcomputing.co.uk/software/pyqt/download!) ;
		\item PIL (\verb!http://www.pythonware.com/products/pil/!) ;
		\item sympy (\verb!http://sympy.org/fr/index.html!) ;
		\item guiqwt/guidata \verb!https://code.google.com/p/{guiqwt,guidata}/!)
		\item pyqtgraph (\verb!http://www.pyqtgraph.org/!) ;
		\item SfePy (\verb!http://sfepy.org/doc-devel/index.html!) ;
		\item FEniCS (\verb!http://fenicsproject.org/!).
	\end{itemize}	
\end{frame}
%%% Local Variables: 
%%% mode: latex
%%% TeX-master: "../java01"
%%% End: 


\subsection{Repr�sentation des entiers : compl�ment � 2}

\begin{frame}
  \frametitle{Contraintes}
  \begin{itemize}
  \item Repr�senter des entiers relatifs
  \item D�terminer si le nombre est positif ou n�gatif
  \item Conserver les propri�t�s de l'addition
  \end{itemize}
$\Rightarrow$ compl�ment � deux
\end{frame}

\begin{frame}
  \frametitle{Compl�ment � deux}
  \begin{itemize}
  \item Bit de poids fort : signe (0$\leadsto$positif ou nul, 1$\leadsto$
    n�gatif)
    \pause
  \item Sur $n$ bits : plus grand entier $2^{n-1}-1$, plus petit $-2^{n-1}$
    \pause
  \item Repr�sentation d'un nombre n�gatif $x$ :
    \begin{itemize}
    \item consid�rer $-x$
    \item inverser chaque bit
    \item ajouter 1
    \end{itemize}
    \pause
  \item Remarque : soit $x$ un entier et $\tilde{x}$ son
    compl�mentaire. $x+\tilde{x}=0$ 
  \end{itemize}
\end{frame}

\begin{frame}
  \begin{exemple}[Repr�sentation de -5 en compl�ment � deux]
    On d�sire coder la valeur -5 sur 8 bits. Il suffit :
    \begin{itemize}
    \item d'�crire 5 en binaire : 00000101 ;
    \item de compl�menter � 1 : 11111010 ;
    \item d'ajouter 1 : 11111011 ;
    \item la repr�sentation binaire de -5 sur 8 bits est 11111011.
    \end{itemize}
  \end{exemple}

    \pause

  \begin{remarque}
    \begin{itemize}
    \item le bit de poids fort est 1 : un nombre n�gatif
    \item 5 + -5 (00000101 + 11111011) donne 0 (retenue de 1)
    \end{itemize}
  \end{remarque}
\end{frame}


\subsection{Repr�sentation des r�els : norme IEEE754}

\begin{frame}
  \frametitle{Objectif}
  \begin{itemize}
  \item Repr�senter des r�els en binaire
  \item N�cessit� d'approximer les nombres
  \item Codage du nombre sur 32 bits en simple pr�cision (64 en double pr�cision)
    \begin{itemize}
    \item signe
    \item valeur
    \item exposant
    \end{itemize}
  \end{itemize}

  $\Rightarrow$ norme IEEE754

  Ex:  $5.25 \leadsto 1.0101*2^2$
\end{frame}


\begin{frame}
  \frametitle{norme IEEE754}

  Repr�sentation (poids fort vers poids faible) :
  \begin{itemize}[<+->]
  \item 1 bit de signe
  \item 8 bits d'exposant pour la simple pr�cision (11 pour la double pr�cision)
  \item 23 bits de mantisse (52 en double pr�cision)
  \item En simple pr�cision : [$\Rightarrow$]seeeeeeeemmmmmmmmmmmmmmmmmmmmmmm 
  \item[=]  $(-1)^s\times (1.M)\times 2^{E-127}$
  \end{itemize}

  ~\\ \pause
  Conditions sur les exposants~:
  \begin{itemize}
  \item 00000000 interdit
  \item 11111111 $\leadsto$ NaN (Not a Number)
  \item[$\Rightarrow$] exposants de -126 � 127 (-1023 � 1024 pour la double pr�cision)
  \end{itemize}
\end{frame}

\begin{frame}
  \frametitle{Exemple}

  \begin{exemple}[Repr�sentation de 525.5 en simple pr�cision]
    \begin{itemize}
    \item<1-8> 525.5 $\xrightarrow{base 2}$ 1000001101.1
    \item<2-8>  \only<2,6->{$1000001101.1 = + 1.0000011011 \times 2^9$}
      \only<3>{$1000001101.1 = \textcolor{red}{+} 1.0000011011 \times 2^9$}
      \only<4>{$1000001101.1 = + 1.0000011011 \times 2^{\textcolor{red}{9}}$}
      \only<5>{$1000001101.1 = +1.\textcolor{red}{0000011011} \times 2^9$}
      \begin{itemize}
      \item<3-8> signe : 0
      \item<4-8> exposant : 127+9=136 $\leadsto$ 10001000
      \item<5-8> mantisse : 0000011011
      \item<6-8>\only<6-8>{[$\Rightarrow$] 01000100000000110110000000000000 }
      \end{itemize}
    \end{itemize}
  \end{exemple}


  \only<7->{
	\begin{remarque}[]
		\begin{itemize}
		\item<7-> Le type \emph{float} en Python utilise une double pr�cision (64 bits).
		\item<8> Les autres bioth�ques utilis�es dans Python (ex. numpy) utilisent autres formats.
	\end{itemize}
	\end{remarque}
}
\end{frame}


\subsection{Repr�sentation des caract�res}

\begin{frame}
  \frametitle{ASCII}

  \only<1,3->{
    \begin{itemize}
    \item<1,3-> ASCII : American Standard Code for Information Interchange
    \item<1,3-> Standard sur 7 bits, �tendu � 8 bits (ex : iso8859-1)
    \item<3-> \emph{Python} utilise la norme \emph{Unicode} : repr�sentation sur 16 bits 
    \end{itemize}
  }
  
	\only<4->{
	\begin{remarque}[Le type caract�re en Python]
		\begin{itemize}
		\item<4-> Contrairement � d'autre langage, il n'existe pas en Python un type sp�cifique pour un caract�re.
		\item<5> Un caract�re n'est rien qu'une cha�ne de caract�res (le type \emph{str}) de longueur 1.
	\end{itemize}
	\end{remarque}
}

	
  \only<2>{
  \begin{tabular}{l|llllllllll}
    &30&40&50&60&70&80&90&100&110&120\\
    \hline
    0:&&(&2&$<$&F&P&Z&d&n&x\\
    1:&&)&3&=&G&Q&[&e&o&y\\
    2:&&*&4&$>$&H&R&\&&f&p&z\\
    3:&!&+&5&?&I&S&]&g&q&\{\\
    4:&"&,&6&@&J&T&\^{}&h&r&|\\
    5:&\#&-&7&A&K&U&\_&i&s&\}\\
    6:&\$&.&8&B&L&V&`&j&t&$\sim$\\
    7:&\%&/&9&C&M&W&a&k&u&DEL\\
    8:&\&&0&:&D&N&X&b&l&v\\
    9:&'&1&;&E&O&Y&c&m&w
  \end{tabular}
}
\end{frame}

%%% Local Variables: 
%%% mode: latex
%%% TeX-master: "../java01"
%%% End: 


\end{document}


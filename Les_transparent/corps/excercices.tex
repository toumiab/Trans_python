% excercice Q/R pour le cours : cours 1
%
\subsection{Excercices}
	\begin{frame}[fragile]
  \frametitle{Excercices}
	Consid�rant le script suivant :
	\begin{exemple}[cartons : \textcolor{green}{vert} = Vrai / \textcolor{red}{rouge} = Faux]
	
\footnotesize{
\begin{semiverbatim}
\only<1,2>{
>>> a = 12.5
>>> b = a
>>> a = a + 0.5 
>>> a
\textcolor{blue}{13 ?}\pause
>>> b
\textcolor{blue}{12.5 ?}
>>>
}\only<3>{
>>> a = "bonjour"
>>> b = a
>>> a + = " tout le monde"
>>> print(b) 
\textcolor{blue}{"bonjour tout le monde" ?}
>>>}\only<4>{
>>> a = "bonjour"
>>> a[0:-1] \textcolor{blue}{\# ==>"b" ?} \pause
>>>}\only<5>{
>>> a = "bonjour"
>>> a[0] = "B"
>>> a \textcolor{blue}{\# ==>"Bonjour" ?} 
>>>}\only<6>{
>>> a, b = [1, 2], ['a']
>>> c = a + b
>>> b.append('b')
>>> c \textcolor{blue}{\# ==> [1, 2, 'a', 'b'] ?}
}\only<7>{
>>> a, b = [[1], 2], ['a']
>>> c = a + b
>>> a[0].append('b')
>>> c \textcolor{blue}{\# ==> [[1,'b'], 2, 'a'] ?}
}
\end{semiverbatim}}	
\end{exemple}
\end{frame}
